\begin{resumo}[Abstract]
 \begin{otherlanguage*}{english}
In most emergency services performed in Brazil's Federal District, there is a risk of cardiac arrest, that require quick action to avoid death. With various points of difficult access in the region , a more effective, and fast, way to answer a emergency call is required. The alternative to the actual model is presented the design of a UAV to aid in these situations, entitled EmerVANT an drone equipped with equipment to carry out the emergency pre-hospital care where there is need for quick access. In addition to the part of emergency care, the project addresses the technical part of the construction of a drone, since the structural part, sensors and electronic control, through external communication to the energy storage. It is possible to see the importance of using the EmerVANT as a tool to achieve a fast and efficient service in order to assist the process of saving lives.

   \vspace{\onelineskip}
 
   \noindent 
   \textbf{Key-words}: UAV. Heart attacks. Respiratory Failure. Emergency assistance.
 \end{otherlanguage*}
\end{resumo}
