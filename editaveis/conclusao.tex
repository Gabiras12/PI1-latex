\chapter{Conclusão}

Com a pesquisa realizada, é possível perceber que a utilização do EmerVant
em casos de emergência contribui para o socorro pré-hospitalar e pode ser determinante para o 
salvamento de vidas humanas, dados os fatores apresentados que impossibilitam a chegada rápida das ambulâncias.

Para o desenvolvimento do EmerVant foram realizadas algumas pesquisas iniciais e foram feitas análises nas áreas 
de estrutura do VANT, controle, comunicação e fonte energética. A análise da fonte energética foi considerada 
um fator muito importante, pois determina a autonomia do drone e a estratégia do voo. 

Para o controle do EmerVant vários sensores foram definidos para que ele tivesse precisão nas informações acerca da localização,
visto que sua atuação é em situações emergenciais e demanda rapidez e agilidade. Para a estrutura do VANT os equipamentos a serem
levados e sua autonomia foram levadas em consideração.

No início do projeto houveram dificuldades com relação a organização da equipe, como por exemplo, nas realizações
de uma atividade por vez com todos os membros, não havendo uma divisão de tarefas. Após algumas reuniões a equipe 
passou a se organizar melhor e dividir as tarefas dentro do projeto. Devido a essa dificuldade também houve um
impacto no cronograma, pois algumas atividades atrasaram e geraram um grande volume de trabalho no período de entrega do relatório.

Nesse relatório foi apresentada apenas uma pesquisa inicial e a definição do escopo acerca da solução a ser desenvolvida. 
Em um segundo momento será realizada uma pesquisa aprofundada para definir materiais, sensores e dimensões estruturais,
projetar o sistema de gestão da informação, a unidade central de processamento e construir protótipos de modo a alcançar os objetivos
propostos.