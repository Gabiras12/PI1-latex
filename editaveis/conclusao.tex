\chapter{Conclusão}

Com a pesquisa realizada, é possível perceber que a utilização do EmerVant
em casos de emergência contribui para o socorro pré-hospitalar e pode ser determinante para o 
salvamento de vidas humanas, dados os fatores apresentados que impossibilitam a chegada rápida das ambulâncias.

Para o desenvolvimento do EmerVant foram realizadas pesquisas aprofundadas e análises nas áreas 
de estrutura do VANT, controle, comunicação e fonte energética. A análise da fonte energética foi considerada 
um fator muito importante, pois foi determinante para a autonomia do drone e a estratégia do voo. 

Para o controle do EmerVant vários sensores foram definidos para que ele tivesse precisão nas informações acerca da localização,
visto que sua atuação é em situações emergenciais e demanda rapidez e agilidade. O controle e a comunicação foram definidas
através de um sistema de controle e comunicação. No qual estão contidos o sistema de comunicação da central de atendimento
com a placa controladora do VANT.

Para a estrutura do VANT aspectos como dimensionamento, carga a ser levada e seus materiais foram levados em consideração.
A partir disso obteve-se um protótipo do EmerVant.

No início do projeto houveram dificuldades com relação a organização da equipe, como por exemplo, nas realizações
de uma atividade por vez com todos os membros, não havendo uma divisão de tarefas. Após algumas reuniões a equipe 
passou a se organizar melhor e dividir as tarefas dentro do projeto. Devido a essa dificuldade também houve um
impacto no cronograma, pois algumas atividades atrasaram e geraram um grande volume de trabalho no período de entrega do primeiro relatório.

Nesse relatório foi apresentado uma solução técnica desenvolvida a partir da solução inicial. Foram apresentados
materiais, sensores, dimensões estruturais, o sistema de gestão da informação, a unidade central de processamento e
os protótipos representativos do VANT a ser construído.
Como trabalho futuro deseja-se obter uma análise dos custos do projeto passando por cada área desenvolvida neste
relatório e a partir disso fazer uma análise de viabilidade do projeto.