\chapter{Conclusão}

Com a pesquisa realizada, é possível perceber  que a utilização do EmerVant em casos de emergência de paradas cardíacas, contribui para o socorro pré-hospitalar de uma maneira mais rápida e pode ser determinante para o salvamento de vidas humanas, dados os fatores apresentados que impossibilitam a chegada rápida  das ambulâncias.

Para o desenvolvimento do EmerVant foram realizadas  pesquisas aprofundadas e análises nas áreas de estrutura do VANT, controle, comunicação e fonte energética. A análise da fonte energética foi considerada  um fator muito importante, pois foi determinante para encontrar a autonomia de vôo, além definir o tempo que o EmerVant levará para chegar até a ocorrência, visto que o tempo de espera é um fator muito decisivo em uma emergência. 
Para  o controle do EmerVant vários sensores foram definidos para que ele tivesse precisão  nas  informações  acerca  da  localização,  visto que  sua  atuação é em situações emergenciais e demanda  rapidez e agilidade. O controle e comunicação serão estabelecidos através de softwares especializados para essa função que será controlado da Central de comandos do EmerVant, por pessoas treinadas e qualificadas.

Para  a estrutura do VANT aspectos como dimensionamento, carga a ser levada e seus materiais foram levados em consideração, e partir disso obteve-se um protótipo do EmerVant. Conforme os cálculos feitos para verificar a tensão de ruptura, concluiu que a tensão de ruptura da fibra de carbono é maior que a tensão máxima ocorrida pelo empuxo de um braço.

No início do projeto houveram  dificuldades com relação a organização da equipe, como por exemplo, nas realizações de uma atividade por vez com todos os membros, não havendo  uma divisão de tarefas. Após algumas reuniões a equipe passou a se organizar melhor e dividir as tarefas dentro do projeto. 
Nesse relatório foi apresentado uma solução técnica desenvolvida a partir da solução inicial, onde foram  apresentados materiais, sensores, dimensões estruturais, o sistema de gestão da informação, a unidade central de processamento e os protótipos representativos do VANT a ser construído. Após a análise técnica, foi possível fazer a viabilidade econômica do projeto, onde pode-se perceber que apesar de um custo um pouco elevado, o projeto é viável e pode ser utilizado como uma alternativa mais acessível em relação ao helicóptero.