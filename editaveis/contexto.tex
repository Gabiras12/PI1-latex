As doenças cardio vasculares fazem parte de condições que predispõe uma pessoa à maiores riscos de 
desenvolver e sofrer ataques do coração e dos vasos sanguíneos. Alguns dos fatores de risco para tais problemas são: 
hábitos alimentares ruins, fumo, obesidade, sedentarismo, diabetes e idade avançada. \cite{ceolin}

As doenças crônicas são responsáveis por mais de 72\% de mortes no Brasil. Onde o governo tem como principal arma a tais patologias, 
a prevenção. Qual age através de diversas ações para prevenir doenças cardíacas e respiratórias pelo Programa Saúde da Família a fim de promover
uma mudança de hábitos a população brasileira, contudo é necessário todo um preparo para o atendimento em casos de ocorrências, devido ao alto grau de 
complexidade do paciente.

Segundo dados provenientes do Ministério da Saúde a população Distrito Federal tem envelhecido, se encontrando em uma pirâmide etária diferente, 
apresentando uma população mais velha e suscetível a doenças crônicas, onde cerca de mais de 33\% possui algum tipo de patologia crônica. \cite{portal}

Essa população é majoritariamente urbana, possuindo cerca de 4,4\% de população rural, devido a sua localização geográfica e o planejamento prévio da cidade. 
Devido a essa grande centralização de pessoas em pequenos espaços, surge uma dificuldade de locomoção no perímetro, principalmente, em horários de pico, 
gerando diversos acidentes entre veículos, de modo que, nas localidades que ocorrem tais ocorrências, o planejamento estrutural de tráfego foi abandonado, 
tornando o socorro ás vitimas mais árduo. \cite{oliveira}

Uma solução para a dificuldade no acesso no local da emergência tem sido a utilização de um Veículo Aéreo Não Tripulado (VANT). 
De acordo com Takei (\citeyear{DEE}), o uso de VANTs tem acarretado no aumento da taxa de sobrevivência e no aumento da eficiência do sistema de pronto-socorro.