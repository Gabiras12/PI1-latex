\subsubsection{Central de Atendimento}
A Central de Atendimento/ Estação Base do VANT, será um local, qual será responsável por controlar toda a operação de atuação do VANT, desde o atendimento da chamada, ao desenvolvimento do plano de voô do mesmo. É responsabilidade do controlador auxiliar no uso do desfibrilador em casos de emergência. Sendo que, tal Base, estará sempre acompanhando o atendimento ao paciente e monitorando o funcionamento do VANT, através da alta gama de sensores e câmeras espalhadas pelo equipamento.

Não há qualquer requerimento especificado para o uso do software de controle, logo um computador padrão com capacidade média de processamento é o suficiente para o uso do software, se vê a importância de destacar que o computador deve possuir uma placa de rede para acesso à internet. Para o computador, foi escolhido um processador Intel I5 e 8GB de memória RAM, com essas configurações é garantido desempenho suficiente com um preço não elevado.  

O atendente que controlará o VANT e atenderá os chamados deve ser um médico socorrista com capacidade de auxiliar o uso do desfibrilador por pessoas leigas. Como a central funcionará 24 horas por dia, se vê necessário o uso de uma rotação de funcionários, levando em conta uma carga horária de 8 horas por dia para cada atendente. Por ser um serviço essencial que não pode ser interrompido, cada atendente deve possuir um substituto hábil de plantão.

Tal Central de Atendimento/Base Estação, apresentará as seguintes características:

\begin{itemize}
  \item 01 - Equipe, composta por 02 pessoas capazes de operar os equipamentos e softwares da Estação Base e do VANT
  \item 02 - Computadores DELL – InspironSmall Desktop \cite{DELL}
  \item 01 Gerador de Energia a Gasolina Portátil \cite{Gerador}
\end{itemize}


\subsubsection{Controlador Pixhawk}
O módulo Pixhawk piloto automático é um sistema muito eficiente em tempo real de operação. 
O software pode ser atualizado com um bootloader USB (gerenciador de boot USB), que é um programa simples com a 
função de acessar o disco do computador e carregar o sistema operacional na memória para assumir o controle do 
equipamento.

Os benefícios do sistema Pixhawk incluem \textit{multithreading}, ou seja, a capacidade de servir mais de um usuário ao 
mesmo tempo, um ambiente de programação Unix/Linux capaz de gerar novas funções de piloto automático, como 
escrita de missões e comportamento de voo. \cite{pix}
 
O módulo Pixhawk flagship pode ser utilizado visando as novas opções de periféricos, como sensor digital de velocidade do ar, o suporte para um indicador LED externo multicor e um magnetômetro externo. Todos os periféricos são automaticamente detectados e configurados. \cite{pix}

\subsubsection{Funcionamento do VANT}

Os VANT's por serem um tipo de veículo de pequeno porte e em conseqüência a miniaturização de seus componentes eletrônicos e seu barateamento, têm-se sua utilização gradativamente aumentada nas mais diversas áreas, como: vigilância, análise ambiental, missões militares e busca e salvamento de pessoas tanto em auto mar, como em áreas de pouco acesso. \cite{Branco}

O VANT projetado,será composto por um complexo sistema integrado, apresentando cinco sub-módulos principais  que trabalham em conjunto afim de obter uma alta plataforma de observação e atuação.

Sendo esses módulos, baseados em \cite{pastor}

\begin{itemize}
	\item Estrutura do VANT – Uma estrutura simples, leve, aerodinamicamente eficiente e com uma plataforma estável, capaz de oferecer uma otimizização na qualidade de voo do mesmo. 
	\item Controlador do VANT – Para realizar todo o processo de controle, comunicação externa e autonomia de voo do VANT, será utilizado o controlador PIXHAWK, da empresa 3D Robotcs, um sistema de computador concebido para coletar informações aerodinâmicas através de um conjunto de sensores(acelerômetros, giroscópios, magnetômetros, sensores de pressão, GPS, etc.), de modo a pilotar automaticamente o VANT.
	\item Caixa de Sensoriamento - Um conjunto de sensores compostos por câmeras de TV, sensores infravermelho, sensores térmicos, etc., para reunir informações que podem ser parcialmente processadas \textit{on-board}, pelo próprio controlador do VANT, ou transmitidas a uma base estação, para posterior análise.
	\item Estação Base – Uma equipe de chão, situada em uma região próxima a área de atuação do VANT, destinada a monitorar o desenvolvimento do voo, acompanhar e orientar o atendimento de emergência e eventualmente operar o VANT, caso ocorra alguma falha técnica. 
	\item Infraestrutura de Comunicação - Uma mistura de mecanismos, equipamentos e técnicas de comunicação (moldens de rádio,satcomm, links de microondas, etc.) que devem garantir um elo contínuo e estável entre o VANT e a estação de base. 
\end{itemize}


% figura

A arquitetura de implementação do VANT utilizará microcontroladores, assim como sensores e módulos de transmissão de dados (Figura \ref{fig:diagrama}).
Os dados transmitidos entre o servidor na central de monitoramento, o VANT e os aplicativos (controle da missão e rastreador) seguirão um determinado padrão. Em caso de perda sinal, com a central de controle, o VANT será programado para que retorne automaticamente para a base de controle.


\subsubsection{Sistema de comunicação via rádio frequência}

Os sistemas de comunicação sem fio são baseados em campos eletromagnéticos, e são ondas que transportam energia de um ponto ao outro, isso possibilita que haja comunicação sem a necessidade de uma conexão física dos fios \cite{VALLE1}. 

Para que haja a propagação do sinal de radiofrequência (RF) é necessário que o sinal seja conduzido por um cabeamento condutor até uma antena que posteriormente irá irradiar através do ar os pulsos eletromagnéticos do sinal \cite{VALLE1}. 

A antena é um componente responsável por converter um sinal, do meio cabeado, em um sinal wireless (sem fio) e vice-versa. Os sinais irradiados no ar livre, em forma de ondas eletromagnéticas, propagam-se em linha reta e em todas as direções \cite{Rappaport2}. Com isso pode-se transmitir qualquer tipo de informação a um receptor remoto apenas utilizando a propagação dessas ondas eletromagnéticas.

% imagem

\subsubsection{Radiofrequências}

O espectro eletromagnético da radiofrequência ocupa as frequências entre os 3 kHz e os 300 GHz. 

A figura X+1 mostra as bandas de radiofrequência, o nome das respectivas e suas aplicações tradicionais.

As ondas de radiofrequência de médias e baixas frequência (MB e ELF) possuem grandes comprimentos de onda, essa característica favorece a difração na atmosfera e garante que os obstáculos de grandes dimensões sejam contornados mais facilmente \cite{Rappaport2}.

Quanto maior a frequência menos é a capacidade de transmitir para distâncias muito longas ao nível da superfície terrestre \cite{VALLE1}. 

\begin{itemize}
	\item As ondas MF são as ondas utilizadas nas estações nacionais, permitindo difundir o som com maior qualidade, embora com menor alcance;
	\item As ondas LF são utilizadas nas estações de rádio que transmitem a nível mundial;
	\item As ondas ELF, que têm frequências extra baixas, são utilizadas quando tem-se grande obstáculos e necessita enviar uma informação. 
	\item As Ondas HF e VHF, possuem pequenos comprimentos de ondas, geralmente sofrendo múltiplas reflexões na ionosfera e na superfície terrestre, não conseguem acompanhar a curvatura da Terra. São utilizadas em comunicações que não exigem grande alcance, mas quando é exigida a alta qualidade de som e imagem.
	\item As Ondas UHF, SHF e EHF são utilizadas para comunicação com satélites, pois o seu comprimento de onda é muito pequeno e ele não sofre quase que refração nenhuma na atmosfera. 
\end{itemize} 

\subsubsection{Agência Nacional de Telecomunicações (ANATEL)}

Para que seja criado um novo link utilizando radiofrequência é necessário consultar a ANATEL, pois ela é responsável por manter o controle de todas as frequências utilizadas nas telecomunicações, para garantir que não exista interferências nas transmissões.

% image

Para poder ter acesso a um canal de frequência é necessário cumprir o regulamento de uso do espectro de radiofrequências. Resolução Nº 259 de 19 de abril de 2001(Em anexo ao relatório);

\subsubsection{Especificação do projeto da comunicação do VANT}

Para a comunicação será utilizado dois canais.

\begin{itemize}
	\item Um canal será responsável apenas pela comunicação dos dados referentes ao deslocamento, ele será para o controle do VANT, então será transmitido informações da trajetória que será seguida para o piloto automático e em caso de emergência o controle manual.
	\item O outro canal será para transmissão dos dados de vídeo e áudio.
\end{itemize}

\subsubsection{A comunicação e transmissão de dados }

O piloto automático guiará o VANT até a emergência. A central receberá a solicitação do veículo e com as orientações do local do acidente, ela irá mapear o percurso a ser seguido para chegar até o destino.  Essas informações referentes ao trajeto e coordenadas de GPS serão passadas de forma wireless serialmente para o sistema e controladores do VANT. Para que isso aconteça, eles estarão conectadas por um canal de comunicação de radiofrequência.

A central necessitará passar informações ao usuário referentes ao atendimento, também precisará ver o estado do paciente e ver se os procedimentos foram realizados conforme as instruções e as informações dos sinais vitais da vítima para entender a real situação dele. Com isso a central receberá um sinal de áudio e vídeo e o VANT receberá apena o sinal de áudio da central. 

\subsubsection{Projeto da comunicação do VANT}
A comunicação da central de atendimento e o VANT será através de um link de radiofrequência, será criado um canal na faixa das VLF (Very Low Frequency).

\begin{itemize}
	\item A escolha dessa faixa de operação VLF foi devido à grande facilidade de propagação desses tipos de onda;
	\item Grande possibilidade de contorno de obstáculos de grandes dimensões;
	\item Baixa suscetibilidade de sofrer interferências;
	\item Propaga-se usando a atmosfera;
	\item Menores chances de perder o link do canal;
	\item A faixa das ELE e VLF estão disponíveis no Brasil. De 0 a 8,5 KHz temos essas duas faixas.
	\item O Canal de comunicação do VANT será nessa faixa de 3KHz a 8,5KHz.
\end{itemize}

\pagebreak
