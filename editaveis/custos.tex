
O EmerVant é um projeto modelo que está designado a atuar na esplanada dos ministérios em Brasília. E por esse motivo haverá produção de apenas um exemplar. O custo da produção em escala é menor que de apenas um exemplar. 

A análise dos custos foi dividida em:
\begin{itemize}
 \item Estrutura do VANT; 
 \item Comunicação;
 \item Controle;
  \item Fonte de energia;
\end{itemize}


\subsection{Custos}
Os custos de cada área e demais custos alocados no projeto, pode ser visto na Tabela \ref{custos}.
\vfill
\begin{table*}[!h]
\centering
    \caption{Custo do projeto}

\begin{tabular}{|p{0.30\linewidth}|p{0.25\linewidth}|}
\hline

Descrição & Valor (em R\$) \\ \hline

Estrutura do VANT & 14.821,13\\ \hline
Fonte energética & 2.189,64 \\ \hline
Controle do VANT & 611,97 \\ \hline
Comunicação do VANT & 63.960,00 \\ \hline
Equipamentos (Desfibrilador e reanimador) & 5.014,00 \\ \hline
Recurso humano alocado por 3 meses & 418.696,5\\ \hline
Custos mensais Central de Controle & 6.667,14\\ \hline
\textbf{Total} & 505.293,24\footnotemark + Custos mensais (6.667,14)\\ \hline

\end{tabular}
    \label{custos}
\end{table*}
\pagebreak

\footnotetext{Valores sujeitos a mudança de acordo com a cotação do dólar.}
\subsection{Análise de viabilidade econômico-financeira do projeto}

  Embora o valor tenha sido elevado, o EmerVant foi considerado um projeto viável financeiramente, visto que
  em comparação com os valores para manutenção de um helicóptero (Tabela \ref{tab:custos}) é uma alternativa
  mais em conta.
