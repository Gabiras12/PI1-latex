\label{escopo}
Para definição do escopo do projeto foram analisados os seguintes aspectos: Local de atuação, localização da apoio, situações de socorro pré-hospitalar, viabilidade dos itens a serem levados pelo VANT, elementos estruturais e planejamento de atuação e estratégia de voo do VANT e o método de comunicação com o usuário.
\begin{description}
  \item[Local de Atuação] \hfill 
  	\begin{itemize}
  		\item Na Esplanada seja por natureza ou circunstância e será acionado apenas quando houver necessidade. Essa escolha foi realizada devido à grande circulação de pessoas e aos eventos esportivos e de entretenimento que ocorrem nesse local.

  	\end{itemize}
  \item[Tipo de emergência] \hfill 
  	\begin{itemize}
  		\item Paradas/Ataques Cardíacos;
		\item Paradas Respiratórias;
  	\end{itemize}
  \item[Equipamentos Utilizados para salvamento] \hfill \\
  	Para definir os itens a serem levados foi realizada uma análise de viabilidade levando em consideração a utilização dos equipamentos por pessoas sem conhecimento em salvamento e a possibilidade de carga útil do VANT.
  	\begin{itemize}
  		\item Kit primeiro socorros;
		\item Desfibrilador Automático;
		\item Reanimador ventilatório manual;
  	\end{itemize}
  \item[Distância de Operação] \hfill 
  	\begin{itemize}
	  
	  \item O VANT operará em um raio máximo de 20 km. Ele partirá de uma base fixa localizada na Rodoviária de Brasília.

	  \item A velocidade máxima do VANT é 70 km/h.  
  	\end{itemize}
  \item[Sinais vitais que serão monitorados] \hfill 
  	\begin{itemize}
	
	  \item Serão monitorados os batimentos cardíacos, os níveis de oxigenação no sangue. Esses sinais são essenciais para verificar o estado do paciente, com eles é possível ter uma conclusão do que aconteceu com a vítima. 
	    \subitem Eletrocardiograma (ECG)
	    \subitem Oxímetro.
		\subsubitem Existem dispositivos que já possuem ECG, oxímetro e termômetro. (Monitor multiparamétrico MX-300)
  	\end{itemize}
  \item[Projeto mecânico estrutural] \hfill 
  
      Para atender todas as especificações e requisitos do projeto é necessário desenvolver o projeto estrutural completo do veículo, para esse projeto foi levado em consideração os possíveis materiais e o tipo de estrutura que melhor atende as necessidades do projeto.
  	O VANT será:
  	\begin{itemize}
  		\item de asa rotatória,pois necessita-se de um voo mais estável e uma maior versatilidade de pousos e decolagens;
		  \subitem Octacóptero.
		\item elétrico, devido a maior eficiência e maiores rendimentos em termos energéticos.
  	\end{itemize}
  \item[Materiais] \hfill 
  
  Os materiais da estrutura foram escolhidos por serem componentes versáteis e com propriedades importantes para a construção da estrutura mecânica do VANT. As propriedades que levaram a escolher esses materiais foram a grande resistência mecânica e baixo peso:
  	\begin{itemize}
  		\item Polímeros compostos;
		\item Fibra de carbono.
  	\end{itemize}
  \item[Controle] \hfill 
  	\begin{itemize}
  		\item O Controle do VANT será através da placa controladora Pixhawk, com ela é possível a implementação do automático e o comando partindo da central de comunicação. A transmissão de dados será através de um link de rádio frequência entre a central de comando e o VANT. A central passará as coordenadas e poderá visualizar todas as partes do atendimento, isso é: 
O deslocamento do VANT até o paciente e os procedimentos realizados. 
  	\end{itemize}
  \item[Comunicação entre equipe de paramédicos e usuário] \hfill 
  	\begin{itemize}
  		\item Câmera e sistema de áudio. Sendo que, o vídeo será apenas para o paramédico da central e a comunicação via áudio para ambos;
  		\item O usuário apenas escutará as instruções, comunicando-se apenas através do áudio.
  	\end{itemize}
  \item[Sensores] \hfill 
  	\begin{itemize}
  		\item Será utilizado um monitor multiparamétrico como o MX-300 para medir todos os sinais vitais necessários;
		\item GPS (u-blox LEA-6H.)- Utilizado para o deslocamento e localização do VANT;
		\item Acelerômetro e Magnetometro (ST Micro LSM303D) - Utilizado para obter uma localização mais apurada;
		\item Giroscópio (ST Micro L3GD20) - Refinamento na localização;
		\item Controladora Pixhawk (PX4).
  	\end{itemize}
  \item[Projeto unidade central de processamento] \hfill 
  	\begin{itemize}
  		\item Controladora Pixhawk (PX4).
  	\end{itemize}
  \item[Conversão e armazenamento de energia] \hfill 
  	\begin{itemize}
  		\item Bateria de polímeros de Lítio (LiPo). A Bateria de LiPo possui uma característica de armazenamento de elevadas capacidades de carga.
  	\end{itemize}
  \item[Estimação de consumo energético e autonomia] \hfill 
  	\begin{itemize}
  		\item Para definir a autonomia será levada em consideração o quanto a bateria pode fornecer em relação ao consumo do motor. 
  	\end{itemize}
\end{description}