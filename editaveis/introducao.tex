\chapter[Introdução]{Introdução}\label{cap1}
	Essa seção aborda sobre o contexto do projeto, o problema a ser resolvido e a metodologia para desenvolvimento do trabalho. 
\section{Contexto}

As doenças cardio vasculares fazem parte de condições que predispõe uma pessoa à maiores riscos de desenvolver e sofrer doenças do coração e dos vasos. Alguns dos fatores de risco para tais problemas são: hábitos alimentares ruins, fumo, obesidade, sedentarismo, diabetes e idade avançada. \cite{ceolin}. Segundo dados do Ministério da Saúde a população Distrito Federal tem envelhecido e mais de 33\% possui algum tipo de doença crônica. \cite{portal}

As doenças crônicas são responsáveis por mais de 72\% de mortes no Brasil, sendo o caminho escolhido pelo governo: a prevenção. Diversas ações têm sido realizadas para prevenir doenças cardíacas e respiratórias pelo Programa Saúde da Família a fim de promover a mudança de hábitos, contudo é necessário o preparo para o atendimento em casos de ocorrências.

A população do Distrito Federal, é majoritariamente urbana, possuindo cerca de 4,4\% de população rural, devido a sua localização geográfica e o planejamento prévio da cidade, é pouco abordada a dificuldade de locomoção no perímetro, contudo, em horários de pico, grandes acidentes entre veículos e em localidades que o planejamento foi abandonado, o socorro é prejudicado. (OLIVEIRA,2007).
\section{Justificativa}

O atendimento pré-hospitalar é o procedimento feito após a ocorrência de um agravo à saúde da vítima \cite{SBC}. A hora de ouro consiste na primeira hora de atendimento à vítima o que pode ser determinante para o salvamento.\cite{PQA} 

As principais ocorrências de atendimento pré-hospitalar de acordo com o Corpo de Bombeiros podem ser vistas na Figura 1. Segundo esses dados aproximadamente 15\% das ocorrências atendidas são os casos de problemas cardíacos, respiratórios e hemorragias.

 \begin{figure}[ht]
	\centering
		\includegraphics[keepaspectratio=true,scale=0.5]{figuras/bombeiros.eps}
	\caption{Gráfico do número de atendimentos pré-hopitalar do corpo de bombeiros. Fonte: Dados do copor de Bombeiros}
\end{figure}

O tempo de atendimento à vítima de PCR (Parada Cardiorrespiratória) é extremamente importante. Receber a RCP (Ressucitação Cardiopulmonar) a tempo pode ser determinante para a sobrevivência do paciente, a cada minuto que passa a chance de sobrevivencia cai de 7 a 10\% \cite{SBC}. 

O DEA (Desfibrilador externo automático) é um equipamento portátil, que interpreta o ritmo cardíaco do paciente, seleciona automaticamente o nível de energia e carrega automaticamente, e tudo que o operador precisa fazer é pressionar o botão de descarga, ele é um equipamento simples que os leigos podem usar com uma simples instrução e é de grande importância no atendimento à PCR, uma vez que grande parte delas ocorre em ambiente extra-hospitalar, como residencia e em lugares públicos \cite{SBC}.

O uso do DEA, assim que disponível, permite maior sucesso no atendimento, pois a maioria das vítimas que tem parada cardíaca em ambiente extra-hospitalar se encontra em FV (Fibrilação Ventricular) \cite{SBC}.

Devido essa importância é possível constatar que há uma necessidade de atendimento mais eficaz nesses casos. Um grande problema que existe em um salvamento é o deslocamento da unidade de salvamento até o local do acidente. Muitas vezes esse caminho está obstruído devido trânsito ou dificuldades de acesso. 

Este projeto tem como objetivo solucionar os problemas referentes a socorros emergenciais, com a intenção de realizar procedimentos que são simples para que não haja a necessidade de um profissional da área e se forem realizados no menor tempo possível, podem aumentar a chance de sobrevivência do paciente. 

O EmerVant vem com a proposta de melhorar esse tempo entre o acidente e a chegada do socorro através do carregamento de alguns equipamentos. De acordo com \cite{DEE}, o uso desse tipo de sistema acarreta no aumento da taxa de sobrevivência, no tempo de resposta reduzido e no aumento da eficiência.

\section{Descrição do Problema}
	A problemática está em deslocar-se em grandes centros urbanos, pois sempre se está sujeito a trânsitos e dificuldades de acesso. Como grande parte das paradas cardiorrespiratórias ocorre em ambiente extra-hospitalar, como residências e centros urbanos \cite{SBC} tem-se a necessidade de agilizar os atendimentos de socorro pré hospitalar. 

\section{Objetivos}
\subsection{Objetivo geral}
	Criar um VANT (veículo aéreo não tripulado) para ajudar a solucionar com mais eficiência o atendimento em situações de emergências no Distrito Federal.
\subsection{Objetivos epecíficos}
\begin{itemize}
  \item Projetar um hexacóptero
  \item Projetar uma unidade central de monitoramento
	\item Projetar sistema de gestão da informação
	\item Definir fonte energética
	\item Definir estratégia do voo
	\item Definir controle do drone
\end{itemize}

\section{Metodologia}
Nessa subseção será abordada a metodologia de trabalho definida para o projeto levando em consideração o seu gerenciamento e a estrutura da equipe.
\subsection{Metodologia de Gerenciamento}

Dentro das possibilidades de gerenciamento de projetos, existem duas metodologias que foram discutidas em grupo que poderiam ser adotadas para o projeto do EmerVANT, a metodologia SCRUM ou a metodologia PMBOK(Project Management Body of Knowledge). Diante do trabalho técnico que deve ser realizado e a quantidade de pessoas envolvidas, o método escolhido foi o do PMBOK porque possibilita e facilita a divisão de trabalho para grupos inexperientes, com menor capacidade variação de mudanças e cria uma cultura de controle que possibilita a elaboração de um projeto altamente crítico. Dentro do PMBOK existem nove áreas de conhecimento (Figura 2) que permitem o gerenciamento do projeto como um todo dentro das quatro estruturas da equipe que serão analisadas posteriormente.

 \begin{figure}[ht]
	\centering
		\includegraphics[keepaspectratio=true,scale=0.5]{figuras/PMBOK.eps}
	\caption{Áreas de conhecimento das fases do PMBOK. Fonte: Autores - Baseado no PMBoK}
\end{figure}

Com o gerenciamento destas nove áreas foi possível definir diretrizes sobre o gerenciamento dos cinco grupos de processos essenciais: iniciação, execução, planejamento, controle e finalização. Os principais documentos criados a partir deste processo serão o TAP(termo de abertura), EAP(Estrutura Analítica de Projeto), plano de gerenciamento de projeto, riscos, comunicações, aquisições e estimativa de custos.

A Figura 3 mostra como funciona o fluxo detalhado das macroatividades do PMBOK e é o processo que está sendo seguindo neste trabalho:

\begin{figure}[ht]
	\centering
		\includegraphics[keepaspectratio=true,scale=0.5]{figuras/pmbok-resumido.eps}
	\caption{Fluxo resumido - PMBOK. Fonte: Mauro Satille. <http://goo.gl/L6R7G9>}
\end{figure}

\subsection{Estrutura da Equipe}
A equipe do projeto, formada por 25 integrantes, foi subdivida em quatro áreas de pesquisas, que são elas: estrutura do VANT, comunicação, controle e fonte energética. As áreas são coordenadas por um subgerente, um para cada área, e ambos são supervisionados e administrados por um gerente geral, Henrique Berilli, e uma gerente de qualidade, Emilie Morais.

A equipe responsável pela estrutura do VANT, gerenciada por Renan Santos, é responsável pelo estudo e seleção dos a serem utilizados, processo de fabricação, sensores, unidade central de processamento e custos do VANT. Esta equipe também é responsável pelo desenho 3D das peças do VANT na ferramenta CATIA.  
	
Coordenada por Leonardo Cambraia, a área responsável pela comunicação, abrangerá o projeto da central de monitoramento, sistema de gestão e informação e custos.
	
A área de controle, gerenciada por Jennifer Cavalcante, estuda a central de controle, o comando do VANT via orientação GPS (Global Positioning System) e pelo link de comunicação VANT - Central de monitoramento.
	
A quarta área, responsável pela fonte energética, e gerenciada por Bárbara Hélen, tem como responsabilidade escolher a fonte de energia, avaliar o consumo energético e autonomia e os custos.
	
A Figura 4 ilustra a estrutura de gerenciamento (hierárquica) do projeto. Foi tomado um cuidado para que nenhum componente da equipe ficasse ocioso. Nota-se que a qualquer momento algum membro poderá ser realocado de modo a melhor suprir as necessidades do projeto.

\begin{figure}[ht]
	\centering
		\includegraphics[keepaspectratio=true,scale=0.5]{figuras/equipe.eps}
	\caption{Estrutura gerencial do projeto.}
\end{figure}

\section{Plano de Comunicação}
O grupo organizou o plano de comunicação em reuniões presenciais semanais. Todas as reuniões são discursões sobre a temática do projeto onde sempre há o relato, através de ata de reunião, dos assuntos e decisões tomadas. 

Reuniões presenciais predefinidas:
\begin{itemize}
	\item Segundas (16h às 18h)
	\item Quartas (16h às 18h)
\end{itemize}

Além das reuniões presenciais, há a comunicação e avisos via grupo nas seguintes ferramentas:
\begin{itemize}
	\item Facebook
	\item WhatsApp
	\item Google Drive
	\item Trello
\end{itemize}

\section{Ferramentas Computacionais}

\textbf{FACEBOOK}:
Foi criado um grupo nesta rede social contendo os 25 membros, denominado de “PI1 Grupo 4 (VANT)”. Neste grupo discutido a possibilidade de futuras reuniões extras, é compartilhado arquivos produzidos fora das reuniões, assim como referências que podem auxiliar no desenvolvimento do projeto. Enquetes são feitas para discutir assuntos gerais pertinentes ao projeto. 

\textbf{WHATSAPP}: 
Neste aplicativo os integrantes comunicam entre si de maneira direta para avisos importantes, pequenas discursões e algumas decisões que não necessitam de uma reunião presencial.

\textbf{GOOGLE DRIVE}: 
Nesta pasta online, os 25 integrantes do grupo tem liberdade para modificar e compartilhar o desenvolvimento do trabalho. Todo o material de pesquisa, atas, relatórios, imagens, questionários e planos de trabalho são disponibilizados, para que todos tenham acesso as informações de cada área, visto que todas as áreas são interligadas.

\textbf{TRELLO}:
Nesta ferramenta organiza-se de tarefas e eventos de uma forma muito dinâmica e funcional, com ela é possível dividir as atividades entre os integrantes, fazer comentários sobre cada função empregada e ter um controle sobre tudo que já foi desenvolvido, os afazeres que estão em fase de desenvolvimento e tudo que está com pendencias.

\textbf{LATEX}:
Nesta ferramenta edita-se os relatórios e resultados do projeto de uma forma mais organizada e com um layout melhor apresentável.






%O mercado referente aos áudiobooks tem crescido
%
%No capítulo \ref{cap2}, é apresentado os objetivos pretendidos para o qual este trabalho %foi motivado.%
%
%No capítulo \ref{cap3}, é apresentada toda a revisão bibliográfica onde foram revistos %monografias, dissertações, livros, publicações em \textit{websites} e especificações %necessários para o entendimento e desenvolvimento do projeto.%
%
%No capítulo \ref{cap4}, serão apresentados todas as etapas realizadas para a especificação %do formato e para o desenvolvimento do Editor bem como as ferramentas utilizadas com %suporte no processo de desenvolvimento e pesquisa.%
%
%No capítulo \ref{cap5}, serão apresentados os resultados obtidos no projeto, mostrando o %arquivo gerado pelo Editor.%
%
%No capítulo \ref{cap6}, 
%E, por fim, no capítulo \ref{cap7},

%Este documento apresenta considerações gerais e preliminares relacionadas 
%à redação de relatórios de Projeto de Graduação da Faculdade UnB Gama 
%(FGA). São abordados os diferentes aspectos sobre a estrutura do trabalho, 
%uso de programas de auxilio a edição, tiragem de cópias, encadernação, etc.

