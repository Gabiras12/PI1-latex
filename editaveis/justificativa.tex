O atendimento pré-hospitalar é o procedimento feito após a ocorrência de um agravo à saúde da vítima \cite{SBC} sendo esse um atendimento de alta complexidade, 
devido ao fato de algumas ocorrências apresentarem riscos de vida ao paciente, que requer um atendimento muito ágil e eficaz, durante a primeira hora após o ocorrido, 
podendo este ser determinante para sobrevivência do mesmo.\cite{PQA} 

As principais ocorrências de atendimento pré-hospitalar de acordo com o Corpo de Bombeiros podem ser vistas na Figura 1. Segundo esses dados aproximadamente 15\% das ocorrências atendidas são os casos de problemas cardíacos, respiratórios e hemorragias.

 \begin{figure}[H]
	\centering
		\includegraphics[keepaspectratio=true,scale=0.5]{figuras/bombeiros.eps}
	\caption[Gráfico do número de atendimentos pré-hopitalar do corpo de bombeiros]{Gráfico do número de atendimentos pré-hopitalar do corpo de bombeiros. Fonte: \cite{bombeiro}}
\end{figure}

O tempo de atendimento à vítima de PCR (Parada Cardiorrespiratória) é extremamente importante. Receber a RCP (Ressucitação Cardiopulmonar) a tempo pode ser determinante para a sobrevivência do paciente, a cada minuto que passa a chance de sobrevivencia cai de 7 a 10\% \cite{SBC}. 

O DEA (Desfibrilador externo automático) é um equipamento portátil, que interpreta o ritmo cardíaco do paciente, seleciona automaticamente o nível de energia e carrega automaticamente, e tudo que o operador precisa fazer é pressionar o botão de descarga, ele é um equipamento simples que os leigos podem usar com uma simples instrução e é de grande importância no atendimento à PCR, uma vez que grande parte delas ocorre em ambiente extra-hospitalar, como residencia e em lugares públicos \cite{SBC}.

O uso do DEA, assim que disponível, permite maior sucesso no atendimento, pois a maioria das vítimas que tem parada cardíaca em ambiente extra-hospitalar se encontra em FV (Fibrilação Ventricular) \cite{SBC}.

Dessa forma, um grande  problema  que existe em um salvamento é o deslocamento da unidade de salvamento até o local do acidente. Muitas vezes esse caminho está obstruído devido a trânsito, dificuldades de acesso como em casos de favelas e invasões ou muitas vezes locais distantes das unidades de pronto-atendimento que estão sobrecarregadas. Essa obstrução de modo geral, causa atraso no socorro.

O  EmerVant  vem  com a  proposta  de otimizar  o tempo  entre  a ocorrência do acidente  e a chegada  do socorro, através do uso de um VANT capaz de carregar de equipamentos que auxiliem o pré-atendimento hospitalar. 

Pinto e Cassemiro (\citeyear{pinto}) afirmam que ``A utilização de Veículos Aéreos Não-Tripulados (VANTs) tem se mostrado uma excelente alternativa, já que dispõe de uma flexibilidade maior e um custo baixo comparado às soluções tradicionais.''

Assim, esse projeto tem como objetivo solucionar os problemas referentes à socorros emergenciais que são causados pela distância entre socorristas e pacientes, realizando através de dois equipamentos e um kit, que irão ser transportados pelo VANT, procedimentos rápidos, onde não será necessária a presença física de um socorrista, e que se forem realizados no menor tempo possível, podem aumentar a chance de sobrevivência do paciente.
