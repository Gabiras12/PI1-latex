Nessa seção será abordada a metodologia de trabalho definida para o projeto levando em consideração o seu gerenciamento e a estrutura da equipe.
\subsection{Metodologia de Gerenciamento}

Dentro das possibilidades de gerenciamento de projetos, existem duas metodologias que foram discutidas em grupo que poderiam ser adotadas para o projeto do EmerVANT, a metodologia SCRUM ou a metodologia PMBOK(Project Management Body of Knowledge). Diante do trabalho técnico que deve ser realizado e a quantidade de pessoas envolvidas, o método escolhido foi o do PMBOK porque possibilita e facilita a divisão de trabalho para grupos inexperientes, com menor capacidade variação de mudanças e cria uma cultura de controle que possibilita a elaboração de um projeto altamente crítico. Dentro do PMBOK existem nove áreas de conhecimento (Figura 2) que permitem o gerenciamento do projeto como um todo dentro das quatro estruturas da equipe que serão analisadas posteriormente.

 \begin{figure}[ht]
	\centering
		\includegraphics[keepaspectratio=true,scale=0.5]{figuras/PMBOK.eps}
	\caption[Áreas de conhecimento das fases do PMBOK]{Áreas de conhecimento das fases do PMBOK. Fonte: Autores (Baseado no PMBoK)}
\end{figure}

Com o gerenciamento destas nove áreas foi possível definir diretrizes sobre o gerenciamento dos cinco grupos de processos essenciais: iniciação, execução, planejamento, controle e finalização. Os principais documentos criados a partir deste processo serão o TAP(termo de abertura), EAP(Estrutura Analítica de Projeto), plano de gerenciamento de projeto, riscos, comunicações, aquisições e estimativa de custos.

A Figura 3 mostra como funciona o fluxo detalhado das macroatividades do PMBOK e é o processo que está sendo seguindo neste trabalho:

\begin{figure}[ht]
	\centering
		\includegraphics[keepaspectratio=true,scale=0.5]{figuras/pmbok-resumido.eps}
	\caption[Fluxo resumido - PMBOK]{Fluxo resumido - PMBOK. Fonte: \cite{pmbok}}
\end{figure}

\subsection{Estrutura da Equipe}
A equipe do projeto, formada por 25 integrantes, foi subdivida em quatro áreas de pesquisas, que são elas: estrutura do VANT, comunicação, controle e fonte energética. As áreas são coordenadas por um subgerente, um para cada área, e ambos são supervisionados e administrados por um gerente geral, Henrique Berilli, e uma gerente de qualidade, Emilie Morais.

A equipe responsável pela estrutura do VANT, gerenciada por Renan Santos, é responsável pelo estudo e seleção dos a serem utilizados, processo de fabricação, sensores, unidade central de processamento e custos do VANT. Esta equipe também é responsável pelo desenho 3D das peças do VANT na ferramenta CATIA.  
	
Coordenada por Leonardo Cambraia, a área responsável pela comunicação, abrangerá o projeto da central de monitoramento, sistema de gestão e informação e custos.
	
A área de controle, gerenciada por Jennifer Cavalcante, estuda a central de controle, o comando do VANT via orientação GPS (Global Positioning System) e pelo link de comunicação VANT - Central de monitoramento.
	
A quarta área, responsável pela fonte energética, e gerenciada por Bárbara Hélen, tem como responsabilidade escolher a fonte de energia, avaliar o consumo energético e autonomia e os custos.
	
A Figura 4 ilustra a estrutura de gerenciamento (hierárquica) do projeto. Foi tomado um cuidado para que nenhum componente da equipe ficasse ocioso. Nota-se que a qualquer momento algum membro poderá ser realocado de modo a melhor suprir as necessidades do projeto.

\begin{figure}[ht]
	\centering
		\includegraphics[keepaspectratio=true,scale=0.5]{figuras/equipe.eps}
	\caption{Estrutura gerencial do projeto.}
\end{figure}

\section{Plano de Comunicação}
O grupo organizou o plano de comunicação em reuniões presenciais semanais. Todas as reuniões são discursões sobre a temática do projeto onde sempre há o relato, através de ata de reunião, dos assuntos e decisões tomadas. 

Reuniões presenciais predefinidas:
\begin{itemize}
	\item Segundas (16h às 18h)
	\item Quartas (16h às 18h)
\end{itemize}

Além das reuniões presenciais, há a comunicação e avisos via grupo nas seguintes ferramentas:
\begin{itemize}
	\item Facebook
	\item WhatsApp
	\item Google Drive
	\item Trello
\end{itemize}

\section{Ferramentas Computacionais}

\textbf{FACEBOOK}:
Foi criado um grupo nesta rede social contendo os 25 membros, denominado de “PI1 Grupo 4 (VANT)”. Neste grupo discutido a possibilidade de futuras reuniões extras, é compartilhado arquivos produzidos fora das reuniões, assim como referências que podem auxiliar no desenvolvimento do projeto. Enquetes são feitas para discutir assuntos gerais pertinentes ao projeto. 

\textbf{WHATSAPP}: 
Neste aplicativo os integrantes comunicam entre si de maneira direta para avisos importantes, pequenas discursões e algumas decisões que não necessitam de uma reunião presencial.

\textbf{GOOGLE DRIVE}: 
Nesta pasta online, os 25 integrantes do grupo tem liberdade para modificar e compartilhar o desenvolvimento do trabalho. Todo o material de pesquisa, atas, relatórios, imagens, questionários e planos de trabalho são disponibilizados, para que todos tenham acesso as informações de cada área, visto que todas as áreas são interligadas.

\textbf{TRELLO}:
Nesta ferramenta organiza-se de tarefas e eventos de uma forma muito dinâmica e funcional, com ela é possível dividir as atividades entre os integrantes, fazer comentários sobre cada função empregada e ter um controle sobre tudo que já foi desenvolvido, os afazeres que estão em fase de desenvolvimento e tudo que está com pendencias.

\textbf{LATEX}:
Nesta ferramenta edita-se os relatórios e resultados do projeto de uma forma mais organizada e com um layout melhor apresentável.
