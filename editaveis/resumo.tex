\begin{resumo}
Tendo em vista que em casos de ataques cardíacos o tempo é extremamente relevante, quando se considera que a chance de sobrevivência a um ataque cardíaco cai 10\%\cite{SBC} a cada minuto, existe a necessidade de um atendimento breve. 
O presente documento tem como objetivo relatar o andamento das atividades desenvolvidas pelo grupo 4, o qual idealizou o projeto de Veículo Aéreo Não Tripulado – VANT,  para assistência emergencial no distrito federal em casos de ataques cardíacos, paradas respiratórias e em alguns acasos de hemorragias externas.
O grupo utilizou de metodologias de gerência de projetos, e os resultados principais advindos dos encontros e discussões da equipe resultam em um desenvolvimento de um possível sistema de apoio a paramédicos.

 \vspace{\onelineskip}
    
 \noindent
 \textbf{Palavras-chaves}: VANT. Ataques cardíacos. Assistência emergencial.
\end{resumo}
