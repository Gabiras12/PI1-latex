\begin{resumo}
Na maioria dos atendimentos emergenciais realizados pelo Corpo de Bombeiros e SAMU no DF existe o risco de paradas cardiorrespiratórias, 
que demandam uma ação rápida para evitar o óbito. Com os vários pontos de difícil acesso na região, um modo mais eficaz para conseguir um atendimento 
rápido se faz necessário. A alternativa apresentada é o projeto de um VANT para auxílio nessas situações, intitulado EmerVANT, um drone octacóptero
equipado com equipamentos para realizar o atendimento pré-hospitalar emergencial onde há a necessidade de acesso rápido. Além da parte de atendimento emergencial, 
o projeto aborda a parte técnica da construção do um drone, desde a parte estrutural, sensores e controlador eletrônico, passando pela comunicação externa até o 
armazenamento energético. É possível visualizar a importância da utilização do EmerVANT como ferramenta para conseguir um atendimento rápido e eficiente, de modo a
auxiliar no salvamento de vidas humanas.


 \vspace{\onelineskip}
    
 \noindent
 \textbf{Palavras-chave}: VANT. Ataques cardíacos. Assistência emergencial.
\end{resumo}
