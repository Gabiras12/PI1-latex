O EmerVant consiste em um sistema de assistência emergencial através da utilização de um VANT.	

O grande potencial dos drones tem sido explorado em inúmeras implementações militares e civis. Entre os vários VANTs, estes de pequena escala são atraentes para o meio acadêmico, devido ao seu pequeno tamanho, as capacidades de voo único, excelente dirigibilidade e baixo custo. \cite{SDM}.

O VANT escolhido para a aplicação foi o multirotor de oito hélices, conhecido como octocóptero. Usando um essa configuração a força de sustentação é melhor dividida entre os múltiplos motores possibilitando carregar uma maior carga útil e tendo uma maior estabilidade.

O \textit{Global Position System} (GPS) é o sensor de maior importância deste projeto, pois as informações fornecidas serialmente por esse componente auxiliam não só na estabilização da aeronave mas no controle e na locomoção do VANT até o local desejado.

Em uma situação normal o desempenho do piloto automático excede a que é controlada por humanos, e mesmo em situações mais desafiadoras sua performance é equivalente.

O projeto mecânico estrutural será constituído de polímeros compostos e fibra de carbono. O VANT será de asa rotativa e utilizará alguns sensores que serão processados por sistemas embarcados dedicados ao voo. 

O controle será feito através de um sistema automático com o uso de GPS, utilizado para deslocamento e localização. Através desse sensor o piloto automático tem acesso às informações em tempo real, utilizando-as ele consegue guiar-se, dessa maneira ele sabe exatamente seu lugar no espaço e sua localização à obstáculos. Para gerar essas informações são necessários: giroscópios, acelerômetro, sensores magnéticos e eletromagnéticos, sensores visuais, sensores ultrassom, infravermelhos, micro-ondas e gamas de rádio. \cite{UDE}

Seu local de atuação será na Esplanada dos Ministérios. Quando acionado a central, que fica em um ponto estratégico da Esplanada, irá receberá a informação referente a necessidade e irá passar de forma serial para o drone as coordenadas referentes  ao local e situação da(s) vítima(s).

O monitoramento do VANT ocorrerá através de um sistema de controle automático por rádio frequência que será monitorado e comandado pela central de monitoramento. Ele terá uma câmera e um sistema de áudio.

O VANT, através da câmera e do sistema de áudio estabelecerá a comunicação entre o usuário,a ambulância e a central de monitoramento. O vídeo será apresentado apenas para o paramédico da central e o áudio para a central e a ambulância. O contato do usuário será apenas com o áudio.

O paramédico dará orientações para quem está com a vítima através do sistema de comunicação do EmerVant. Com a câmera os paramédicos da central de comando poderá visualizar a situação, dar instruções para utilizar o desfibrilador ou reanimador ventilatório manual. 
